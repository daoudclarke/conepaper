\section{Introduction}

Distributional approaches to representing the semantics of natural language
are an established and important component in the computational
linguist's toolbox. In these approaches, words are typically
represented as vectors, leading to a variety of ways of measuring the semantic 
similarity of two words based on these vectors~\cite{Lee:99}. 

Despite the widespread use of distributional approaches, we still lack a comprehensive theory
of natural language meaning that is vector based. While there are been recent work  
looking at how vector based 
representations of meaning can be composed in a
similar way to how fragments of logic are composed in Montague 
semantics~\cite{Clark:08,Grefenstette:11},  
little attention has been paid to how 
distributional semantics relates to logical semantics. 

Researchers in distributional semantics typically focus
on similarity, whilst logical semantics places an emphasis on
equivalence, contradiction and entailment. Whilst semantic similarity can be
relevant to many problems arising in natural language processing applications, 
tasks such as textual entailment recognition, summarisation and ontology learning
deal with entailment. Similarity measures are not sufficient to
represent entailment: they are generally symmetric, whilst entailment
is antisymmetric and transitive.

In this paper we examine the proposition of \newcite{Clarke:07} that
entailment can be described by a partial ordering on the vector
space. Whilst many vector spaces can be considered to be partially
ordered vector spaces merely by the existence of a distinguished
basis, in this paper we investigate the possibility of learning a new
partial ordering directly from data. We describe an algorithm which
does this and demonstrate its effectiveness on some simple datasets.

We also describe how this technique could potentially be applied to
three different problems: recognising textual entailment, ontology
learning, and compositionality in vector space semantics. As this is
preliminary work, we leave the implementation of these proposals to
further work.
