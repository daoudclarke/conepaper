\section{Introduction}

Distributional approaches to representing the semantics of natural language
are an established and important component in the computational
linguist's toolbox. In these approaches, words are typically
represented as vectors, leading to a variety of ways of measuring the semantic 
similarity of two words based on these vectors~\cite{Lee:99}. 

Despite the widespread use of distributional approaches, we still lack a comprehensive theory
of natural language meaning that is vector based. While there are been recent work  
looking at how vector based 
representations of meaning can be composed in a
similar way to how fragments of logic are composed in Montague 
semantics~\cite{Clark:08,Grefenstette:11},  
little attention has been paid to how 
distributional semantics relates to logical semantics. 

Researchers in distributional semantics typically focus
on similarity, whilst logical semantics places an emphasis on
equivalence, contradiction and entailment. Whilst semantic similarity can be
relevant to many problems arising in natural language processing applications, 
tasks such as textual entailment recognition, summarisation and ontology learning
deal with entailment. Similarity measures are not sufficient to
represent entailment: they are generally symmetric, whilst entailment
is antisymmetric and transitive.

There has been interest in number of rather different notions of entailment, including logical entailment, lexical entailment and textual entailment (further details are provided in Section~\ref{sec:background}). In order to be able to capture such diverse definitions of entailment, our goal has been to devise a method that is able to express these different conceptions of entailment in a  distributional semantics setting.

Distributional generality~\cite{Weeds:04,Clarke:07} provides a starting point, identifying one particular way that the distributional vectors associated with two words can be used to define an ordering on words. This relationship is based on the extent to which one word's distributional contexts subsume those of another word.

In this paper, we generalise this notion of distributional generality based on the proposition of \newcite{Clarke:07} that
entailment can be described by a partial ordering on the vector
space. Whilst many vector spaces can be considered to be partially
ordered vector spaces merely by the existence of a distinguished
basis, in this paper we investigate the possibility of learning a new
partial ordering directly from data. 

Given two vectors $\mathbf{u}$ and $\mathbf{v}$, we can say that $\mathbf{u}$ is less than $\mathbf{v}$, $\mathbf{u}\le\mathbf{v}$, when the vector obtained by subtracting $\mathbf{v}$ from $\mathbf{u}$, $\mathbf{u}-\mathbf{v}$, is `positive'.  We can therefore formulate  any given partial ordering (and thereby some particular notion of entailment) by specifying which of these difference vectors are to be considered `positive' for that particular partial ordering. 

A characterisation such as this can be captured with a matrix. This
matrix can be defined in such a way that the result of taking its
product with some vector $\mathbf{u}-\mathbf{v}$, the result (a vector) is positive if
and only if all of its components are non-negative. �The $i$th row of
the matrix corresponds to defines a half-space according to whether
its dot product with a vector is positive or negative. The $i$th
component of the resulting vector is positive when $\mathbf{u}-\mathbf{v}$ is contained
in this half-space. Thus, the space containing all positive vectors is
an intersection of half spaces, or more abstractly, a \textbf{cone}.


Given this way of formulating entailment via a partially ordered vector space, we are able to learn a particular entailment relation by learning one of these matrices. We describe an algorithm which
does this and demonstrate its effectiveness on some simple datasets.

We also describe how this technique could potentially be applied to
three different problems: recognising textual entailment, ontology
learning, and compositionality in vector space semantics. 
