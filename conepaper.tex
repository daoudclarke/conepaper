%Bismillahi-r-Rahmani-r-Rahim
\documentclass[11pt]{article}
\usepackage{acl2013}
\usepackage{times}
\usepackage{url}
\usepackage{latexsym}
\usepackage{subcaption}

\usepackage{algorithm}
\usepackage{algpseudocode}

\usepackage{color}
\usepackage{graphicx}
\usepackage{csvsimple}

\usepackage{amsthm}
\usepackage{amssymb}
\usepackage{amsmath}

\newtheorem{definition}{Definition}
\newtheorem{proposition}{Proposition}
\newtheorem{example}{Example}

\newcommand{\R}{\mathbb{R}}

\DeclareMathOperator*{\argmin}{\arg\!\min}

\title{A data driven approach to modelling  entailment in distributional semantics}
\author{First Author \\
  Affiliation / Address line 1 \\
  Affiliation / Address line 2 \\
  Affiliation / Address line 3 \\
  {\tt email@domain} \\\And
  Second Author \\
  Affiliation / Address line 1 \\
  Affiliation / Address line 2 \\
  Affiliation / Address line 3 \\
  {\tt email@domain} \\}
\date{}

\begin{document}

\maketitle

\begin{abstract}
This paper concerns the question of how to capture the notion of semantic entailment in a distributional semantics setting. We present an approach that considers entailment to involve a partial ordering on a vector space, and show how a partial ordering on finite dimensional spaces can be expressed as matrices. We give an algorithm for learning such matrices, and therefore partial orderings, from positive and negative examples, and apply the algorithm to a number of datasets.
\end{abstract}

\section{Introduction}

Distributional approaches to representing the semantics of natural language
are an established and important component in the computational
linguist's toolbox. In these approaches, words are typically
represented as vectors, however we still lack a comprehensive theory
of natural language meaning that is vector based.

Part of the problem is arguably that we don't know how to relate
distributional semantics to logical semantics. Recent work has looked
at how vector based representations of meaning can be composed in a
similar way to how fragments of logic are composed in Montague
semantics.

However there is another problem that has until now not received a lot
of attention. Researchers in distributional semantics typically focus
on similarity, whilst logical semantics places an emphasis on
equivalence, contradiction and entailment. Whilst similarity is
sufficient for many tasks in natural language processing, tasks such
as textual entailment recognition, summarisation and ontology learning
deal with entailment. Similarity measures are not sufficient to
represent entailment: they are generally symmetric, whilst entailment
is antisymmetric and transitive.

In this paper we examine the proposition of \newcite{Clarke:07} that
entailment can be described by a partial ordering on the vector
space. Whilst many vector spaces can be considered to be partially
ordered vector spaces merely by the existence of a distinguished
basis, in this paper we investigate the possibility of learning a new
partial ordering directly from data. We describe an algorithm which
does this and demonstrate its effectiveness on some simple datasets.

We also describe how this technique could potentially be applied to
three different problems: recognising textual entailment, ontology
learning, and compositionality in vector space semantics. As this is
preliminary work, we leave the implementation of these proposals to
further work.


\section{Background}
\label{sec:background}

In this section, we  briefly discuss three distinct notions of entailment that have played a role
in natural language semantics, then go on to consider in more detail what has been referred to as
distributional entailment.

\subsection{Entailment}

There are at least three notions of entailment described in the
literature. The oldest and best defined is \textbf{logical entailment}, which
is (in general) a transitive and antisymmetric relation (i.e.~a
preorder) on a formal language (that of the logic in question).

\textbf{Lexical entailment}~\cite{Geffet:05} is that described by taxonomies, 
such as the IS-A
or hypernymy relation in WordNet. For example, a \emph{cat} is an
\emph{animal}, so \emph{cat} lexically entails \emph{animal}. 
Like logical entailment, this relation is normally
assumed to be a preorder, and often it is also assumed to be a partial
order.

Finally, \textbf{textual entailment}~\cite{Dagan:05} is a recent 
innovation intended to
characterise a problem that lies at the heart of many of the tasks 
that arise in natural language processing applications, including
document summarisation, information retrieval and machine
translation. A sentence $T$ (the ``text'') is said to entail a
sentence $H$ (the ``hypothesis'') if a reader will generally infer
that $H$ is true given that $T$ is true. This is clearly a more fuzzy
relation than logical entailment, leading to disagreement among readers
as to whether this condition holds in certain cases.

\subsection{Distributional Entailment}

\newcite{Weeds:04} proposed the notion of \textbf{distributional
  generality}, that words with more general meanings will tend to
occur in a wider range of contexts. \newcite{Clarke:07} formalised
this idea using a partially ordered vector space.

\begin{definition}[Partially ordered vector space]
  A partially ordered vector space $V$ is a real vector space together
  with a partial ordering $\le$ such that:\\[5pt]
  \indent if $\mathbf{u}\le\mathbf{v}$ then $\mathbf{u} + \mathbf{w} \le\mathbf{v} + \mathbf{w}$\\[5pt]
  \indent if $\mathbf{u} \le\mathbf{v}$ then $\alpha \mathbf{u} \le \alpha \mathbf{v}$
  \vspace{0.1cm}\\[5pt]
  for all $\mathbf{u},\mathbf{v},\mathbf{w} \in V$, and for all $\alpha \ge 0$. 
  
  Such a partial
  ordering is called a \textbf{vector space order} on $V$. An element
  $u$ of $V$ satisfying $u \ge 0$ is called a \textbf{positive
    element}; the set of all positive elements of $V$ is denoted
  $V^+$. If $\le$ defines a lattice on $V$ then the space is called a
  \textbf{vector lattice} or \textbf{Riesz space}.
\end{definition}

\begin{example}[Lattice operations on $\R^n$]
  \label{example:finite}
  A vector lattice captures many properties that are inherent in real
  vector spaces when there is a \emph{distinguished basis}. In $\R^n$,
  given a specific basis, we can write two vectors $\mathbf{u}$ and $\mathbf{v}$ as
  sequences of numbers: $\mathbf{u} = (u_1,u_2,\ldots u_n)$ and $\mathbf{v} =
  (v_1,v_2,\ldots v_n)$. This allows us to define the lattice
  operations of meet $\land$ and join $\lor$ as
\[\begin{array}{l}
\mathbf{u}\land \mathbf{v} =\\
(\min(u_1,v_1),\min(u_2,v_2),\ldots \min(u_n,v_n))\\[5pt]
\mathbf{u}\lor \mathbf{v} = \\
(\max(u_1,v_1),\max(u_2,v_2),\ldots \max(u_n,v_n))
\end{array}\]
These are the component-wise minimum and maximum, respectively. The partial
ordering is then given by $\mathbf{u} \le\mathbf{v}$ if and only if $\mathbf{u} \land \mathbf{v} = u$, or
equivalently $u_i \le v_i$ for all $i$.
\end{example}


%\section{Motivation}

\section{Learning Lattice Orderings}

In this section we show how lattice orderings on finite dimensional
spaces can be described as matrices, and give our algorithm for
learning such matrices from data.

\begin{proposition}[Partial Orderings Defined by Linear Operators]
  Let $A$ be a vector space and $B$ a partially ordered vector space,
  and let $M$ be a linear function from $A$ to $B$. Define an ordering
  on $A$ by $u\le v$ if and only if $Mu \le Mv$. Then $A$ is a
  partially ordered vector space under this ordering.
\end{proposition}

\begin{proof}
If $u \le v$ then $M(u + z) = Mu + Mz \le Mv + Mz = M(v +
z)$ so $u + z \le v + z$. Similarly, $M(\alpha u) = \alpha Mu
\le\alpha Mv = M(\alpha v)$ so $\alpha u \le \alpha v$ for
$\alpha \ge 0$.
\end{proof}

In particular, if we take $A = \R^n$ and $B = \R^m$ then $M$ is an $n$
by $m$ matrix. If we take the ordering for $A$ from Example
\ref{example:finite} then $u\le v$ if and only if $(Mu)_i \le (Mv)_i$
for all $i$.

\subsection{Learning Algorithm}

\begin{algorithm}
\caption{Cone Learning by Gradient Descent}\label{algorithm:cone}
\begin{algorithmic}
  \Procedure{LearnCone}{$V, \mathbf{c}, d$}
  \State $M \gets$ \Call{RandMatrix}{d}
  \State $\alpha \gets \alpha_0$
  \Loop
    \State $M' \gets$ \Call{Gradient}{$V, \mathbf{c}, M$}
    \Loop
      \State $N \gets M - \alpha M'$
      \State $N' \gets$ \Call{Gradient}{$V, \mathbf{c}, N$}
      \If{$\|N'\|_1 < \|M'\|_1$}
        \State \textbf{break}
      \EndIf
      \State $\alpha \gets \alpha/2$
      \If{$\alpha < \alpha_t$}
        \State \textbf{return} $M$
      \EndIf
    \EndLoop
    \State $M \gets N$
    \State $\alpha \gets 1.2\alpha$
  \EndLoop
  \EndProcedure
  \\
  \Procedure{Gradient}{$V, \mathbf{c}, M$}
  \State $A \gets MV$
  \State $B \gets$ \Call{Project}{$A, \mathbf{c}$}
  \State \textbf{return} $(A - B)V^T$
  \EndProcedure
  \\
  \Procedure{Project}{$U, \mathbf{c}$}
  \For{each column vector $\mathbf{u}_i$ in $U$}
    \If{$c_i = 1$}
      \State $\mathbf{u}_i \gets (\mathbf{u}_i)^+$
    \Else
      \State $m \gets \min(\mathbf{u}_i)$
      \If{$m > -0.1$}
        \State $j' \gets \argmin_j{(\mathbf{u}_i)_j}$
        \State $(\mathbf{u}_i)_j \gets -0.1$
      \EndIf
    \EndIf
  \EndFor
  \State \textbf{return} $U$
  \EndProcedure
\end{algorithmic}
\end{algorithm}

In this section we describe our learning algorithm (Algorithm
\ref{algorithm:cone}). Our approach is inspired by that of
\newcite{Hoyer:04}, who describes an approach for learning sparse
non-negative matrix factorisations based on gradient descent. Although
our goal and implementation is unique, we are able to adapt the
gradient descent technique to our ends. Our algorithm is implemented
in Python and is open sourced under the MIT license.\footnote{Location
  withheld to preserve anonymity.}

Let $\mathbf{v}$ be an instance vector and $c \in \{0,1\}$ the class value of
the instance, where 0 indicates the instance is not positive, and 1
indicates that it is positive. Our goal is to learn a matrix $M$ such
that $M\mathbf{v} \ge 0$ if and only if $c = 1$, for as many instance vectors
as possible.

Let $V$ be the matrix whose columns are the instance vectors in the
training data, and let $\mathbf{c}$ be the binary vector of
corresponding class values.

We also need to fix the parameter $d$, the dimensionality of the space
describing the partial ordering. Then $M$ is a $d$ by $n$ matrix,
where $n$ is the number of instance features, i.e.~the dimensionality
of the instance vectors $\mathbf{v}$. We initialise this to a random
matrix with values in $[-1,1]$, and then begin the loop in which the
gradient descent occurs. The gradient $M'$ is computed by projecting
$MV$ such the resulting matrix satisfies the goal of making all
positive instances in the training data positive ($\mathbf{u}^+$ is
the vectors with all negative components of $\mathbf{u}$ set to zero)
and all non-positive instances are not positive. The gradient is then
estimated by multiplying the difference between $MV$ and its
projection with $V^T$.

The parameter $\alpha$ is tuned using a heuristic from the
implementation of \newcite{Hoyer:04}, and determines how far we move
in the direction of the gradient in each loop. We initialise $\alpha$
to $\alpha_0$ and stop when $\alpha$ is smaller than the threshold
$\alpha_t$.

\subsection{Avoiding Local Minima}

We found in our initial experiments, particularly with low-dimensional
data, that the gradient descent algorithm gets stuck in local
minima. To mitigate this, we do a few iterations of the gradient
descent algorithm from a number of different initial matrices, and
choose the one that works the best on the training data. We then do a
gradient descent until convergence on this best performing matrix.

\subsection{Handling Noisy Data}

We also found in testing that the algorithm could not cope well with
noisy data; in particular if a data point is marked as positive but is
actually in the negation of the cone, this prevents the algorithm from
finding the optimal solution. To get around this, we relaxed the
projection algorithm so that it only affects those that are already
closest to being correct. This effectively gives us automatic outlier
detection, where the proportion of outliers that should not be
adjusted in the projection is specified as a parameter.



\section{Experiments}

\subsection{Toy Data}

\subsection{MlData.org Datasets}

\subsection{Hypernymy Detection}

We built a dataset using distributional information of nouns together
with taxonomy information from WordNet.
\begin{enumerate}
\item We identified ``pseudo-monosemous'' nouns in WordNet by
  selecting those whose most most frequent sense accounted for more
  than 80\% of occurrences (we used the frequency information included
  with WordNet), giving us 2,351 nouns.
\item We removed the term \emph{entity} as this is indirectly related
  to every noun, and we didn't want our dataset to be biased by this
  edge case.
\item We identified the set $H$ of all pairs of terms $(t_1, t_2)$
  such that $t_2$ is an indirect hypernym of $t_1$, using the most
  frequent sense of each term. This gave us 4,794 pairs; the first ten
  are shown in Table \ref{table:pairs}.
\item We chose the same number of pairs at random from the remainder
  of possible pairs to form non-positive examples
\item For each term, we built a vector using dependency relations,
  extracted using (parser?) from (corpus?).
\item We constructed the dataset using the vector $u_2 - u_1$ where
  $u_i$ is the vector obtained for term $t_i$, and where the class was
  positive if the indirect hypernymy relation holds for the pair.
\item Reduce the dimensionality using sparse random projections.
\end{enumerate}

\begin{table}
\begin{center}
\begin{minipage}{3cm}
abandon trait\\
abortion event\\
abscess knowledge\\
absurdity nonsense\\
abyss object\\
abyss location\\
academy institution\\
academy school\\
academy group\\
academy grouping\\
\end{minipage}
\end{center}
\caption{The first ten entailment pairs obtained from WordNet.}
\label{table:pairs}
\end{table}

\begin{table}
\begin{center}
\csvautotabular{datasets.csv}
\end{center}
\caption{Datasets from mldata.org used in our experiments, with number
  of instances, features and classes in each dataset.}
\end{table}


\section{Results}


\begin{table*}
\begin{center}
\csvautotabular{toy.csv}
\end{center}
\caption{Accuracy results on toy data for each classifier. The first
  number in the dataset name indicates the dimensionality of the data,
  and the second the number of vectors generating the cone used
  to construct the data.}
\label{table:toy}
\end{table*}

\begin{table*}
\begin{center}
\csvautotabular{results-wn.csv}
\end{center}
\caption{Accuracy results for the datasets constructed using WordNet.
  The last number indicates the dimensionality of the
  dataset after random projection. Three methods were used to construct the term vectors:
  ``random'' datasets used random term vectors of the specified
  dimensionality, ``raw'' datasets used the raw feature frequency and
  ``mi'' datasets used pointwise mutual information, discarding
  features with a negative value.}
\label{table:wn}
\end{table*}

\begin{figure}
\begin{center}
\includegraphics[width=\linewidth]{dimensions.pdf}
\end{center}
\caption{Accuracy (y axis) of each dataset with the cone classifier as
  the dimensionality (x axis) varies. This data is derived from the
  nested cross validation performed for the parameter search.}
\label{fig:dim}
\end{figure}

Table \ref{table:toy} shows results for the artificially constructed
datasets for the three classifiers. Errors shown in all tables are
estimates of error in the mean from the five folds in the
cross-validation. It is clear that both decision trees and cones are
able to learn the cone structure well in this idealised situation, and
in general they outperform the linear SVMs.  As expected, linear SVMs
are unable to learn the cone structure completely, since they are
restricted to learning an half-space (although one that need not pass
through the origin).

We can see that our learning algorithm still needs perfecting; the
poor performance on the two dimensional data is due to problems with
gradient descent falling into local minima.

Table \ref{table:wn} shows accuracy results for the WordNet
datasets. Since the datasets are balanced, baseline accuracy is 0.5;
all classifiers do significantly better than this. Although all
datasets should have a cone structure, as they are constructed from
the partial ordering of WordNet, we wanted to confirm that we can
learn something about entailment from the context vectors
themselves. To do this, we included a baseline in which the term
vectors are generated at random, instead of being derived from corpus
data. With 100 dimensions, this baseline is surprisingly hard to beat,
with 73\% accuracy from both the SVM and cone classifiers. With term
vectors constructed using raw frequency counts, no classifier improves
on this baseline. Weighting and filtering the features using pointwise
mutual information improves the situation, and both the support vector
machine and cone algorithms outperform the random vector baseline.

Figure \ref{fig:dim} shows the performance of the cone classifier on
each dimension in the nested cross validation used for the parameter
search. The best performing dataset peaks at three dimensions, showing
that we are really making use of the ability to describe a cone. Other
datasets perform best with only one dimension, meaning their
expressivity is equivalent to an unbiased linear SVM, which learns a
half-space whose plane passes through the origin. Poor performance at
higher dimensions are likely due to limitations of our learning
algorithm and the difficulty of learning higher-dimensional cones, and
we hypothesise that this is the reason that the SVM outperforms the
cone classifier.

The cone classifier may be the most useful for some applications,
since it gives us a partial ordering, whereas the orderings obtained
from SVMs and decision trees are not guaranteed to have the properties
of a partial ordering. For example, in ontology learning, the
classifications have satisfy the requirements of the ontology; for
cone classifiers these will be satisied automatically, whereas for
other classifiers, some process would be needed to solve conflicts
between the classifications and the ontology requirements.



\section{Related Work}

\newcite{Weeds:04}, who introduced the notion of distributional
generality, considered the simpler task of determining the entailment
direction, given two words for which it is known that entailment
exists, but the direction of entailment is not know. They evaluated
two unsupervised ways of doing this on hyponym-hypernym pairs
extracted from WordNet. Firstly, they found that the frequency of the
term predicted the entailment direction 70\% of the time, with the
more general term being the more frequent. Secondly, they looked at
the precision and recall of the features associated with two terms,
where the features of one term are used to predict the occurence of
the feature with the other term; this gives a measure of which term
occurs in a wider range of contexts. They found this predicted the
correct entailment direction 71\% of the time. Along this strain of
unsupervised approaches, there have been a number of investigations
into directional or asymmetric distributional similarity measures and
their application
\cite{Geffet:05,Bhagat:07,Szpektor:08,Clarke:09,Kotlerman:10}.

The Stanford WordNet project \cite{Snow:04} expands the WordNet
taxonomy by analysing large corpora to find patterns that are
indicative of hyponymy. For example, the pattern ``$\mathit{NP}_X$ and
other $\mathit{NP}_Y$'' is an indication that $\mathit{NP}_X$ is a
$\mathit{NP}_Y$, i.e.~that $\mathit{NP}_X$ is a hyponym of
$\mathit{NP}_Y$. They use machine learning to identify other such
patterns from known hyponym-hypernym pairs, and then use these
patterns to find new relations in the corpus. They enforce the
transitivity relation of the taxonomy by only searching over valid
taxonomies, and evaluating the likelihood of each taxonomy given the
available evidence \cite{Snow:06}. A cone classifier would obviate the
need for this search, since only valid taxonomies would be produced by
the classifier.

\newcite{Kobayashi:08} propose a method of classification using
cones. In their method, the cone is generated by the positive class,
and classification is performed by projecting a given vector onto the
cone and measuring the angle between the original and projected
vectors. An instance whose angle is less than a learnt threshold is
assumed to be positive. Their approach works since their goal is
different to ours: they are interested only in classification, not
learning a partial ordering. Moreover, since they use the generating
cone, it is not clear how sensitive their technique is to noise in the
training data, whereas our approach explicitly accounts for this.



\section{Conclusion and Further Work}

We have discussed the use of vector space orderings to describe
entailment and shown how such an ordering can be described as a
matrix. We have demonstrated that these orderings can be learnt from
data, and described our algorithm for doing so.

We evaluated our approach on both toy datasets and datasets
constructed using the partial ordering of the WordNet taxonomy, and we
showed that our algorithm is successfully able to learn this ordering.

In future work, we hope to improve our algorithm further, in
particular to improve its robustness with respect to local minima. We
also plan to investigate the application of cone learning to textual
entailment.

One of the motivations for this work is the question of how to compose
distributional representations of meaning. Our hope is that we can use
the techniques described here to learn lattice orderings on vector
space representations of strings of words that will allow them to be
related to one another in way that more closely resembles the
entailment of logical semantics.


\newpage

\bibliographystyle{acl}
\bibliography{contexts}


\end{document}