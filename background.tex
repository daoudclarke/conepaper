\section{Background}
\label{sec:background}

In this section, we  briefly discuss three distinct notions of entailment that have played a role
in natural language semantics, then go on to consider in more detail what has been referred to as
distributional entailment.

\subsection{Entailment}

There are at least three notions of entailment described in the
literature. The oldest and best defined is \textbf{logical entailment}, which
is (in general) a transitive and antisymmetric relation (i.e.~a
preorder) on a formal language (that of the logic in question).

\textbf{Lexical entailment}~\cite{Geffet:05} is that described by taxonomies, 
such as the IS-A
or hypernymy relation in WordNet. For example, a \emph{cat} is an
\emph{animal}, so \emph{cat} lexically entails \emph{animal}. 
Like logical entailment, this relation is normally
assumed to be a preorder, and often it is also assumed to be a partial
order.

Finally, \textbf{textual entailment}~\cite{Dagan:05} is a recent 
innovation intended to
characterise a problem that lies at the heart of many of the tasks 
that arise in natural language processing applications, including
document summarisation, information retrieval and machine
translation. A sentence $T$ (the ``text'') is said to entail a
sentence $H$ (the ``hypothesis'') if a reader will generally infer
that $H$ is true given that $T$ is true. This is clearly a more fuzzy
relation than logical entailment, leading to disagreement among readers
as to whether this condition holds in certain cases.

\subsection{Distributional Entailment}

\newcite{Weeds:04} proposed the notion of \textbf{distributional
  generality}, that words with more general meanings will tend to
occur in a wider range of contexts. \newcite{Clarke:07} formalised
this idea using a partially ordered vector space.

\begin{definition}[Partially ordered vector space]
  A partially ordered vector space $V$ is a real vector space together
  with a partial ordering $\le$ such that:\\[5pt]
  \indent if $x \le y$ then $x + z \le y + z$\\[5pt]
  \indent if $x \le y$ then $\alpha x \le \alpha y$
  \vspace{0.1cm}\\[5pt]
  for all $x,y,z \in V$, and for all $\alpha \ge 0$. 
  
  Such a partial
  ordering is called a \textbf{vector space order} on $V$. An element
  $u$ of $V$ satisfying $u \ge 0$ is called a \textbf{positive
    element}; the set of all positive elements of $V$ is denoted
  $V^+$. If $\le$ defines a lattice on $V$ then the space is called a
  \textbf{vector lattice} or \textbf{Riesz space}.
\end{definition}

\begin{example}[Lattice operations on $\R^n$]
  \label{example:finite}
  A vector lattice captures many properties that are inherent in real
  vector spaces when there is a \emph{distinguished basis}. In $\R^n$,
  given a specific basis, we can write two vectors $u$ and $v$ as
  sequences of numbers: $u = (u_1,u_2,\ldots u_n)$ and $v =
  (v_1,v_2,\ldots v_n)$. This allows us to define the lattice
  operations of meet $\land$ and join $\lor$ as
\[\begin{array}{l}
u\land v =\\
(\min(u_1,v_1),\min(u_2,v_2),\ldots \min(u_n,v_n))\\[5pt]
u\lor v = \\
(\max(u_1,v_1),\max(u_2,v_2),\ldots \max(u_n,v_n))
\end{array}\]
These are the component-wise minimum and maximum, respectively. The partial
ordering is then given by $u \le v$ if and only if $u \land v = u$, or
equivalently $u_i \le v_i$ for all $i$.
\end{example}
